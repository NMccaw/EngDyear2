\documentclass{article}
\usepackage{parskip}
\usepackage{graphicx}
\usepackage{float}
\usepackage{amsmath}
\usepackage{xcolor}
\usepackage{caption}
\usepackage{subcaption}
\usepackage{geometry}
\usepackage{listings}

\geometry{a4paper, portrait, margin=1.25in}
\title{PPTC Prop Tutorial Notes}
\author{Nicholas McCaw}
\begin{document}
	\maketitle
	\section{Dimensions}
	Dimensions are defined for each property. Particularly in the p, U, k and omega files in the 0 folder. The dimensionSet is set in a square bracket with 7 entries. Each entry is defines the unit. i.e [mass length time temp quantity current luminous]. The number inside the square bracket defines the order of the unit.
	
	Example: U is m/s and will therefore be defined as [0 1 -1 0 0 0 0].
	
\end{document}